\documentclass[12pt, pdftex]{article}
\usepackage{amsmath}
\usepackage{amssymb}
\DeclareMathOperator*{\argmin}{arg\,min}

\begin{document}
\section*{Graphical Models Energy Minimization}
\noindent
\textsl{[Amir]}

\noindent
\textsl{2015-05-18}

Consider the following \texttt{MAP} problem for the graphical model
$(\mathcal{G},\mathcal{X_V},\theta)$:
\begin{equation}
x^{\star}=\argmin_{x\in\mathcal{X_V}} E(x) = \argmin_{x\in\mathcal{X_V}}
\sum_{v\in\mathcal{V}} \theta_v(x_v) + \sum_{uv \in \mathcal{E}}
\theta_{uv}(x_u,x_v)
\end{equation}
By defining vector $\mu$, this problem can be formulated as the following integer linear
program:
\begin{equation}
\label{eq:ilp}
\begin{aligned}
& \underset{x}{\text{min}}
& & \sum_{v\in\mathcal{V}} \sum_{x_v \in \mathcal{X_V}} \theta_v(x_v)\mu_v(x_v)
+ \sum_{uv\in\mathcal{E}} \sum_{x_u,x_v \in \mathcal{X_UV}} \theta_{uv}(x_u,x_v)
\mu_{uv}(x_u,x_v)\\
& \text{s.t}
& & \sum_{x_v} \mu_v(x_v) = 1, v\in\mathcal{V},\\
&&& \sum_{x_v} \mu_{uv}(x_u, x_v) = \mu_u(x_u) \\
&&& \sum_{x_u} \mu_{uv}(x_u, x_v) = \mu_v(x_v) \\
&&& \mu \geq 0 \\
&&& \mu_v \in \{0,1\}^I
\end{aligned}
\end{equation}

All but integrity constraints of \ref{eq:ilp} can be represented in a form
\begin{equation}
\begin{aligned}
A\mu = b \\
\mu \geq 0
\end{aligned}
\end{equation}

Hence, they form a \textit{polytope} in the space $\mathbb{R}^I$ which is call
\textit{local polytope} and will be denoted by $\mathcal{L}$. So, equation
\ref{eq:ilp} can be represented by
\begin{equation}
\label{eq:primal_short}
\begin{aligned}
\underset{\mu \in \mathcal{L} \cap \{0,1\}^I}{\text{min}} <\theta,\mu> .
\end{aligned}
\end{equation}
Ommitting integrality constraint leads to its \textit{local polytope (LP)
relaxation} often called also \textit{linear programming relaxation}, gives us:
\begin{equation}
\label{eq:primal_relaxed}
\underset{\mu \in \mathcal{L}}{\text{min}} <\theta,\mu> .
\end{equation}
where $\mu$ will be called \textit{primal} w.r.t the problem
\ref{eq:primal_short} or \ref{eq:primal_relaxed}.

khaabam gereft. Tangy is low now.
\end{document}
